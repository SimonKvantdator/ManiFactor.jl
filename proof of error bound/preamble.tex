%! TEX root = /home/simon/Documents/dagbok_2021-2022/main.tex

%%%%%%%%%%% LANGUAGE %%%%%%%%%%%

% For correct hyphenation in swedish
\usepackage[T1]{fontenc}

% For interpreting non-ASCII characters
\usepackage[utf8]{inputenc}

% International language support
% Fetches language from documentclass options. Most other packages do this as well
\usepackage{babel}


%%%%%%%%%%% FORMAL STUFF %%%%%%%%%%%

% Smaller margins
\usepackage[margin=2.5cm]{geometry}

% Indent at new paragraph
\setlength{\parindent}{4ex}

% Clickable urls
% \usepackage[hyphens]{url} % Does not want to be loaded after physics package

% Fancy chapter headers
\usepackage{titlesec}
\titleformat{\chapter}{\normalfont\huge}{\thechapter.}{20pt}{\huge\it}

% Dates & time
\usepackage[yyyymmdd]{datetime} % Useful when referencing websites
\renewcommand{\dateseparator}{-} % ISO 8601 date format

% What to display in table of contents
\setcounter{tocdepth}{1}
\setcounter{secnumdepth}{2}

% Lists
\usepackage{enumerate} % Determines the style in which the counter is printed
\usepackage{enumitem} % Provides user control over the layout of the three basic list environments

% Citing & bibliography
\usepackage{csquotes} % For \enquote command for proper quotation marks, biblatex uses this if it is loaded
% \usepackage[backend=biber, style=numeric, sorting=none]{biblatex}
% \bibliography{bibliography} % A file named bibliography.bib containing the bibTeX entries should be placed beside main.tex
% If not used, the preceding two lines should be commented out. Otherwise LaTeX will complain of an empty bibliography


%%%%%%%%%%% GRAPHICS %%%%%%%%%%%

\usepackage{graphics,color,xcolor}

% Figures
\usepackage{epsfig} % Solves some problems in \includegraphics{<.eps-file>}
\usepackage{graphicx} % More options for \includegraphics
\usepackage{wrapfig} % Figure environment that lets text wrap around figure
\usepackage{float} % Improved figure placement options
\usepackage{caption} % More options for \caption
\usepackage{subcaption} % Subfigures

% PGF/TikZ
% They are two "low level" languages for producing graphics
\usepackage{tikz}
\usepackage{pgf, pgfplots}
\pgfplotsset{compat=1.15}

% For the Aldus leaf (\ding{166})
\usepackage{pifont}


%%%%%%%%%%% PHYSICS %%%%%%%%%%%

% SI units
\usepackage{siunitx}
\DeclareSIUnit\clight{\text{$c$}} % redefine from c_0 to c
\DeclareSIUnit\byte{B}

% Physics macros
\usepackage{physics} % Defines lots of nice commands like \derivative, \norm, \evaluated, etc. It is recommended to use these as much as possible for nice spacing and readable LaTeX code.
\usepackage{braket} % Defines \bra, \ket, \braket, and \set
\usepackage{slashed} % For Feynman slash notation
% \usepackage{simpler-wick} % Wick contractions (may require sty-file)
% \usepackage[compat=1.1.0]{tikz-feynman} % Feynman diagrams (has to be compiled with LuaTeX)
\usepackage{tensor} % Covariant index notation


%%%%%%%%%%% CODING %%%%%%%%%%%

% For nice code insertions
\usepackage{listings}
\definecolor{codegreen}{rgb}{0,0.6,0}
\definecolor{codegray}{rgb}{0.5,0.5,0.5}
\definecolor{codepurple}{rgb}{0.58,0,0.82}
\definecolor{backcolour}{rgb}{0.95,0.95,0.92}
\lstdefinestyle{mystyle}{
    backgroundcolor=\color{backcolour},   
    commentstyle=\color{codegreen},
    keywordstyle=\color{magenta},
    numberstyle=\tiny\color{codegray},
    stringstyle=\color{codepurple},
    basicstyle=\ttfamily\footnotesize,
    breakatwhitespace=false,         
    breaklines=true,                 
    captionpos=b,                    
    keepspaces=true,                 
    numbers=left,                    
    numbersep=5pt,                  
    showspaces=false,                
    showstringspaces=false,
    showtabs=false,
    tabsize=4
}
\lstset{style=mystyle}


%%%%%%%%%%% MATHEMATICS %%%%%%%%%%%

% AMS packages
\usepackage{amsmath, amsfonts, amsthm, amssymb}

% Theorem and proof environments
% Beware, using package parskip will fuck with the spacing of these environments
\iflanguage{swedish}{
    \newtheorem{theorem}{Sats}
    \newtheorem*{theorem*}{Sats}
    \newtheorem{proposition}{Proposition}
    \newtheorem*{proposition*}{Proposition}
    \newtheorem{corollary}{Följdsats}[theorem]
    \newtheorem*{corollary*}{Följdsats}
    \newtheorem{lemma}{Lemma}
    \newtheorem*{lemma*}{Lemma}
	\newtheorem{conjecture}{Conjecture}
	\newtheorem*{conjecture*}{Conjecture}
    \theoremstyle{definition}
    \newtheorem{definition}{Definition}
    \newtheorem*{definition*}{Definition}
}{}
\iflanguage{english}{
    \newtheorem{theorem}{Theorem}
    \newtheorem*{theorem*}{Theorem}
    \newtheorem{proposition}{Proposition}
    \newtheorem*{proposition*}{Proposition}
    \newtheorem{corollary}{Corollary}[theorem]
    \newtheorem{corollary*}{Corollary}
    \newtheorem{lemma}{Lemma}
    \newtheorem*{lemma*}{Lemma}
	\newtheorem{conjecture}{Förmodan}
	\newtheorem{conjecture*}{Förmodan}
    \theoremstyle{definition}
    \newtheorem{definition}{Definition}
    \newtheorem*{definition*}{Definition}
}{}

% Better version of the \not command
\usepackage{cancel}

 % Does polynomial division for you
\usepackage{polynom}

% Some of my own math commands that I haven't found great packages for yet
%! TEX root = /home/simon/Documents/Dagbok_MPPHS_2020-2021/main.tex
% Vectors are upright boldface. I think this definition is better than the physics package's \vectorbold.
\let\Vec\undefined % We use \vec w/ lowercase v
\renewcommand*{\vec}[1]{{\boldsymbol{\mathrm{#1}}}}

% Bar, tilde, and hat that scales with what is under them. Basically I just want these to have consistent names
\let\mathbar\overline
\let\mathtilde\widetilde
\let\mathhat\widehat

% Redefine \exp
% Errors occur if this definition is made before some of the packages are loaded
\let\oldexp\exp
\newcommand*{\Exp}[1]{\oldexp{#1}}
\renewcommand{\exp}[1]{\mathrm{e}^{#1}}

% Main number systems
\newcommand{\naturals}{\mathbb{N}}
\newcommand{\integers}{\mathbb{Z}}
\newcommand{\rationals}{\mathbb{Q}}
\newcommand{\reals}{\mathbb{R}}
\newcommand{\complexnumbers}{\mathbb{C}}

% Some of my own shorthands for correct spacing in math environments
\def\divides{\mid} % Proper spacing of vertical bar in division x|y
\def\from{\colon} % Proper spacing of colon in functions f:A→ B
\newlength\mylen % Isomorphic \mapsto
\settowidth\mylen{$\longleftrightarrow$}
\newcommand{\mapsbetween}{\longleftrightarrow\kern - 0.5\mylen\vline height 1.2ex depth -0.0pt\kern0.5\mylen}
\newcommand{\suchthat}{\qq{s.th.}}
\def\definedas{\coloneqq}
\def\defines{\eqqcolon}

\newcommand*{\transpose}[1]{{#1}^{\!\mathsf{T}}}
\renewcommand*{\complement}[1]{{#1}^{\mathsf{C}}}
\newcommand{\conjugate}[1]{\mathbar{#1}}
% \newcommand*{\conjugate}[1]{{#1}^*}
\newcommand*{\hermitianconjugate}[1]{{#1}^\dag}
\newcommand*{\inverse}[1]{{#1}^{-1}}

\newcommand*{\closure}[1]{\mathbar{#1}} % Closure of a set
\def\union{\cup}
\newcommand{\Union}{\bigcup\limits}
\def\intersection{\cap}
\newcommand{\Intersection}{\bigcap\limits}

% Lie-groups & algebras, i.g. SU(n)
\newcommand*{\algebra}[2]{{\mathfrak{\MakeLowercase{#1}}}{\left(#2\right)}}
\newcommand*{\group}[2]{{\mathrm{\MakeUppercase{#1}}}{\left(#2\right)}}

% Fundamental operators
\def\sDiv{\mathscr{D}}
\def\sTwist{\mathscr{T}}
\def\sCurl{\mathscr{C}}
\def\sCurlDagger{\mathscr{C}^\dagger}

% Symmetric multiplication
\newcommand*{\SymMult}[2]{\overset{#1, #2}{\odot}}

% SymH
\DeclareMathOperator*{\Sym}{Sym}



%%%%%%%%%%% MISCELLANEOUS %%%%%%%%%%%

% In-pdf comments through \todo command
% \setlength{\marginparwidth}{2cm} % Silence warning about margin size
% \usepackage{todonotes}
% The todonotes package is very slow, the following setup is a bit uglier but saves alot of compilation time. It may be also used with the "inline" option.
\usepackage{xstring}
\usepackage[]{marginnote}
\newcommand{\todo}[2][]{%
	\IfStrEqCase{#1}{%
        {inline}{{\color{red}#2}}%
	}[%
		\marginnote{\color{red}#2}%
	]
}

% Clickable links and refs
\usepackage[hidelinks]{hyperref} 

% Cleverref automatically detects if you are referencing a figure, table, or equation etc
% Cleverref has to be loaded last I think, after babel and hyperref etc
\usepackage[noabbrev, nameinlink]{cleveref}
\crefname{equation}{}{}
\iflanguage{swedish}{ % Tell cleverref to use Oxford comma
	\newcommand{\creflastconjunction}{, och\nobreakspace}
}{}
\iflanguage{english}{
	\newcommand{\creflastconjunction}{, and\nobreakspace}
}{}


% Tag only referenced equations (this is a bad package, as it requires etextools, which is buggy and abandoned by its author)
% \expandafter\def\csname ver@etex.sty\endcsname{3000/12/31}
% \let\globcount\newcount
% \usepackage{autonum}

% Intervals on the real line
\let\interval\undefined % To avoid name conflict with etextools
\usepackage{interval}
\intervalconfig{soft open fences}

% For writing \overset{text}&{=} in align environment
\usepackage{aligned-overset} 

% Generate tables from csv files
\usepackage[]{csvsimple} 
